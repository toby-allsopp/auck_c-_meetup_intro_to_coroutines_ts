% Created 2016-06-02 Thu 16:16
\documentclass[aspectratio=43]{beamer}
\usepackage[utf8]{inputenc}
\usepackage[T1]{fontenc}
\usepackage{listings}
\usepackage{booktabs}
\usepackage{sourcecodepro}
%\usepackage{sourcesanspro}
\usepackage{arev}
\usepackage{pgfpages}

%\usetheme{Berkeley}
%\usecolortheme{seahorse}
%\usefonttheme{professionalfonts}
\usefonttheme[onlysmall]{structurebold}
\useinnertheme{}
\useoutertheme{}
\setbeamertemplate{navigation symbols}{\insertframenumber/\inserttotalframenumber}
%\setbeameroption{show notes on second screen}

\date{Auckland C++ Meetup\\9 August 2017}
\title{Introduction to the Coroutines TS}
\subtitle{Part I}
\author[Toby Allsopp]{Toby Allsopp\\\texttt{toby@mi6.gen.nz}}

\hypersetup{
 pdfauthor={},
 pdftitle={},
 pdfkeywords={},
 pdfsubject={},
 pdfcreator={},
 pdflang={English}}

\AtBeginSection[]{
  \begin{frame}
  \vfill
  \centering
  \begin{beamercolorbox}[sep=8pt,center,shadow=true,rounded=true]{title}
    \usebeamerfont{title}\insertsectionhead\par%
  \end{beamercolorbox}
  \vfill
  \end{frame}
}

\lstset{
    frame=single,
    backgroundcolor=,
    language=[11]C++,
    basicstyle=\small\ttfamily,
    commentstyle=\color{gray},
    firstnumber=auto,
    identifierstyle=\color{black},
    keywordstyle=\bfseries,
    numberstyle=\color{purple},
    stringstyle=\color{red},
    captionpos=top,
    numbers=left,
    numberstyle=\Tiny\color{gray},
    numbersep=5pt,
    showspaces=false,
    showstringspaces=false,
    showtabs=true,
    tabsize=4,
    emph={co_yield,co_await,co_return},
    emphstyle=\bfseries
}
\lstset{basicstyle=\ttfamily}

\begin{document}

\frame{\titlepage}

\begin{frame}
  \frametitle{Overview}
  \tableofcontents
\end{frame}

\section{The Technical Specification}

\begin{frame}
  \frametitle{What's a technical specification?}
  \pgfimage[width=\textwidth]{wg21-timeline-2017-07b.png}
  \note{A TS is like a feature branch of the IS}
\end{frame}

\begin{frame}
  \frametitle{The coroutines technical specification}
  \begin{itemize}
  \item \emph{Programming Languages — C++ Extensions for Coroutines}
  \item TS ``draft'': \url{http://wg21.link/n4680} (published 2017-07-30)
  \item Championed by Gor Nishanov (MSFT)
  \item Voted for publication at the July ISO C++ committee meeting
    \begin{itemize}
    \item All national body comments have been addressed
    \item No guarantee it will be included in C++20
    \end{itemize}
    \pause
  \item Implemented in MSVC and Clang
    \begin{itemize}
    \item Visual Studio 2015 (minor differences wrt. TS)
    \item Clang trunk (will be version 5.0 I think)
    \end{itemize}
  \end{itemize}
\end{frame}

\note{We'll go into the contents of the TS soon, but first we need to answer...}

\section{What's a Coroutine?}

\begin{frame}
  \frametitle{What are these things?}
  \begin{itemize}
  \item Lots of subtly different things in many different languages are called ``coroutines''
  \item It's important not to get confused by experience with other kinds of coroutines
    \pause
  \item Generalization of functions --- coroutines \emph{are} functions
  \item Can be suspended and resumed
  \end{itemize}
\end{frame}

\note{From now on, we'll focus on the kind of coroutines defined in the TS.}

\section{What are they good for?}

\begin{frame}
  \frametitle{Canonical coroutine applications}
  \begin{itemize}
  \item Generators
  \item Async tasks
  \item Async generators
  \end{itemize}
\end{frame}

\begin{frame}[fragile]
  \frametitle{Example: printing primes}
  \note{Imagine we want to print successive prime numbers until told to stop}
  Here's a utility class we'll use on future slides; nothing to do with
  coroutines yet.
\begin{lstlisting}
struct prime_tester {
  vector<int> primes;

  bool test(int n) {
    if (none_of(primes.begin(), primes.end(),
          [n](int p) { return n%p == 0; })) {
      primes.push_back(n);
      return true;
    }
    return false;
  }
};
\end{lstlisting}
\end{frame}

\begin{frame}[fragile]
  \frametitle{Example: printing primes}
  \note{Imagine we want to print successive prime numbers until told to stop}
  Callback (push) version:
  \begin{lstlisting}
template <typename Print>
void print_primes(Print print) {
  prime_tester tester;
  for (int n = 2; ; ++n)
    if (tester.test(n))
      if (!print(n))
        break;
}
\end{lstlisting}
\end{frame}
\begin{frame}[fragile]
  \frametitle{Primes generator}
  Generator (pull) version:
  \begin{lstlisting}

generator<int> primes() {
  prime_tester tester;
  for (int n = 2; ; ++n)
    if (tester.test(n))
      co_yield n;

}
\end{lstlisting}
\end{frame}

\begin{frame}[fragile]
  \frametitle{Primes generator}
  To use the callback version:
\begin{lstlisting}
print_primes([](int n) {
  cout << n << "\n";
  return n < 100;
});
\end{lstlisting}
    \pause
  To use the generator, just iterate over it:
\begin{lstlisting}
for (int n : primes()) {
  cout << n << "\n";
  if (n >= 100) break;
}
\end{lstlisting}
\end{frame}

\begin{frame}[fragile]
  \frametitle{Primes generator}
\begin{lstlisting}
generator<int> primes() {
  prime_tester tester;
  for (int n = 2; ; ++n)
    if (tester.test(n))
      co_yield n;
  }
}
\end{lstlisting}
  \begin{itemize}
  \item \lstinline~co_yield~ is the first of three new keywords
  \item \lstinline~generator<T>~ is a class with \lstinline~begin()~ and
    \lstinline~end()~ members
  \item \lstinline~generator~ is not provided by the TS but you can make it
    yourself
  \end{itemize}
\end{frame}

\begin{frame}[fragile,t]
  \frametitle{Example: reading from a socket}
  Synchronously:
\begin{lstlisting}
     string  connect_and_read(string host,
                              int port) {
  string result;
  auto socket =          connect(host, port);
  array<char, 1024> buffer;
  while (int nread =
                    socket.read(buffer))
    result.append(buffer, nread);
     return result;
}
\end{lstlisting}
\end{frame}
\begin{frame}[fragile,t]
  \frametitle{Example: reading from a socket}
  Asynchronously:
\begin{lstlisting}
task<string> connect_and_read(string host,
                              int port) {
  string result;
  auto socket = co_await connect(host, port);
  array<char, 1024> buffer;
  while (int nread =
           co_await socket.read(buffer))
    result.append(buffer, nread);
  co_return result;
}
\end{lstlisting}
  \begin{itemize}
  \item Two more new keywords: \lstinline~co_await~ and \lstinline~co_return~
  \item \lstinline~task~ could be \lstinline~std::future~ or any other class that has the
    coroutine customization points implemented
  \end{itemize}
\end{frame}

\begin{frame}[fragile]
  \frametitle{Example: asynchronous prime stream}
\begin{lstlisting}
struct async_pt {
  static task<async_pt> create();
  task<bool> test(int n);
};
\end{lstlisting}
\begin{lstlisting}
async_generator<int> primes() {
  auto tester = co_await async_pt::create();
  for (int n = 2; ; ++n)
    if (co_await tester.test(n))
      co_yield n;
}
\end{lstlisting}
\begin{lstlisting}
for co_await (int n : primes()) {
  cout << n << "\n";
  if (n >= 100) break;
}
\end{lstlisting}
\end{frame}

\begin{frame}[fragile]
  \frametitle{Example: asynchronous prime stream}
\begin{lstlisting}
struct async_pt {
  socket_t socket;
  static task<async_pt> create() {
    auto s = co_await connect("host", 1234);
    co_return async_pt{s};
  }
  task<bool> test(int n) {
    co_await socket.write(to_string(n) + "\n");
    byte b = co_await socket.read();
    co_return b != 0;
  }
};
\end{lstlisting}
\end{frame}

\begin{frame}[fragile]
  \frametitle{Example: asynchronous prime stream}
\begin{lstlisting}
async_generator<int> primes() {
  auto tester = co_await async_pt::create();
  for (int n : ints(2))
    if (co_await tester.test(n))
      co_yield n;
}
\end{lstlisting}
  \begin{itemize}
  \item \lstinline~async_generator<T>~ is a hypothetical class representing an
    asynchronous stream
  \item can use both \lstinline~co_await~ and \lstinline~co_yield~
  \end{itemize}
\end{frame}

\begin{frame}[fragile]
  \frametitle{Example: asynchronous prime stream}
  The interface of \lstinline~async_generator~ is a lot like
  \lstinline~generator~ but \lstinline~begin()~ and \lstinline~operator++()~ are
  asynchronous.
\begin{lstlisting}
template <typename T>
class async_generator {
  task<iterator> begin();
  iterator end();

  class iterator {
    task<void> operator++();
    T& operator*() const;
  };
};
\end{lstlisting}
\end{frame}

\begin{frame}[fragile]
  \frametitle{Example: asynchronous prime stream}
  What about this \lstinline~for co_await~ thing?
\begin{lstlisting}
for co_await (int n : primes()) {
  cout << n << "\n";
  if (n >= 100) break;
}
\end{lstlisting}
  \pause
  It's equivalent to:
\begin{lstlisting}
auto&& ag = primes();
for (auto it = co_await ag.begin();
     it != ag.end();
     co_await ++it) {
  int n = *it;
  cout << n << "\n";
  if (n >= 100) break;
}
\end{lstlisting}
\end{frame}

\section{How can I use this stuff today?}

\begin{frame}[fragile]
  \frametitle{Using coroutines today}
  \begin{itemize}
  \item Compilers and standard libraries
    \begin{itemize}
    \item MSVC from Visual Studio 2015 Update 3 (\texttt{/await})
    \item Clang with libc++ recent trunk builds (\texttt{-stdlib=libc++ -fcoroutines-ts})
    \end{itemize}
  \item Supporting libraries
    \begin{itemize}
    \item cppcoro (\url{https://github.com/lewissbaker/cppcoro})
    \item range-v3 (\url{https://github.com/ericniebler/range-v3})
    \end{itemize}
  \end{itemize}
\end{frame}

\end{document}

% Local Variables:
% TeX-command-extra-options: "-shell-escape"
% End:
