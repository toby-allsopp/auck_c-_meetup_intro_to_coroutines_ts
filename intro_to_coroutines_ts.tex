% Created 2016-06-02 Thu 16:16
\documentclass[aspectratio=43]{beamer}
\usepackage[utf8]{inputenc}
\usepackage[T1]{fontenc}
\usepackage{listings}
\usepackage{booktabs}
\usepackage{sourcecodepro}
%\usepackage{sourcesanspro}
\usepackage{arev}
\usepackage{pgfpages}

%\usetheme{Berkeley}
%\usecolortheme{seahorse}
%\usefonttheme{professionalfonts}
\usefonttheme[onlysmall]{structurebold}
\useinnertheme{}
\useoutertheme{}
\setbeamertemplate{navigation symbols}{\insertframenumber/\inserttotalframenumber}
%\setbeameroption{show notes on second screen}

\date{Auckland C++ Meetup\\9 August 2017}
\title{Introduction to the Coroutines TS}
\subtitle{Part I}
\author[Toby Allsopp]{Toby Allsopp\\\texttt{toby@mi6.gen.nz}}

\hypersetup{
 pdfauthor={},
 pdftitle={},
 pdfkeywords={},
 pdfsubject={},
 pdfcreator={},
 pdflang={English}}

\AtBeginSection[]{
  \begin{frame}
  \vfill
  \centering
  \begin{beamercolorbox}[sep=8pt,center,shadow=true,rounded=true]{title}
    \usebeamerfont{title}\insertsectionhead\par%
  \end{beamercolorbox}
  \vfill
  \end{frame}
}

\lstset{
    frame=single,
    backgroundcolor=,
    language=[11]C++,
    basicstyle=\small\ttfamily,
    commentstyle=\color{gray},
    firstnumber=auto,
    identifierstyle=\color{black},
    keywordstyle=\bfseries,
    numberstyle=\color{purple},
    stringstyle=\color{red},
    captionpos=top,
    numbers=left,
    numberstyle=\Tiny\color{gray},
    numbersep=5pt,
    showspaces=false,
    showstringspaces=false,
    showtabs=true,
    tabsize=4,
    emph={co_yield,co_await,co_return},
    emphstyle=\bfseries
}
\lstset{basicstyle=\ttfamily}

\begin{document}

\frame{\titlepage}

\begin{frame}
  \frametitle{Overview}
  \tableofcontents
\end{frame}

\section{The Technical Specification}

\begin{frame}
\end{frame}

\begin{frame}
  \frametitle{What's a Technical Specification?}
  \pgfimage[width=\textwidth]{wg21-timeline-2017-07b.png}
  \note{A TS is like a feature branch of the IS}
\end{frame}

\begin{frame}
  \frametitle{The Coroutines Technical Specification}
  \begin{itemize}
  \item \emph{Programming Languages — C++ Extensions for Coroutines}
  \item PDTS Draft: \href{http://wg21.link/n4663}{[N4663]} (published 2017-03-25)
  \item TS Draft: \href{http://wg21.link/n4680}{[N4680]} (not yet published)
  \item Championed by Gor Nishanov (MSFT)
  \item Voted for publication at the July ISO C++ committee meeting
    \begin{itemize}
    \item All national body comments have been addressed
    \item No guarantee it will be included in C++20
    \end{itemize}
    \pause
  \item Implemented in MSVC and Clang
    \begin{itemize}
    \item Visual Studio 2015 (minor differences wrt. TS)
    \item Clang trunk (will be version 5.0 I think)
    \end{itemize}
  \end{itemize}
\end{frame}

\note{We'll go into the contents of the TS soon, but first we need to answer...}

\section{What's a Coroutine?}

\begin{frame}
  \frametitle{What are these things?}
  \begin{itemize}
  \item Lots of subtly different things in many different languages are called ``coroutines''
  \item It's important not to get confused by experience with other kinds of coroutines
    \pause
  \item Generalization of functions --- coroutines \emph{are} functions
  \item Can be suspended and resumed
  \end{itemize}
\end{frame}

\begin{frame}
  \frametitle{Prior Art}
  \begin{itemize}
  \item Boost.Coroutine
  \item Boost.Coroutine2
  \end{itemize}
\end{frame}

\begin{frame}
  \frametitle{Other Proposals}
  \begin{itemize}
  \item call/cc
  \end{itemize}
\end{frame}

\note{From now on, we'll focus on the kind of coroutines defined in the TS.}

\section{What are they good for?}

\begin{frame}
  \frametitle{Canonical coroutine applications}
  \begin{itemize}
  \item Generators
  \item Futures
  \item Async streams
  \end{itemize}
\end{frame}

\begin{frame}[fragile,fragile]
  \frametitle{Example: Printing Primes}
  \note{Imagine we want to print successive prime numbers until told to stop}
\begin{lstlisting}
struct prime_tester {
  vector<int> primes;

  bool test(int n) {
    if (none_of(primes.begin(), primes.end(),
          [n](int p) { return n%p == 0; })) {
      primes.push_back(n);
      return true;
    }
    return false;
  }
}
\end{lstlisting}
\end{frame}

\begin{frame}[fragile,fragile]
  \frametitle{Example: Printing Primes}
  \note{Imagine we want to print successive prime numbers until told to stop}
  \begin{lstlisting}
template <typename ShouldStop, typename Print>
void print_primes(ShouldStop should_stop,
                  Print print) {
  prime_tester tester;
  for (int n = 2; ; ++n) {
    if (should_stop()) break; // <-- UGLY
    if (tester.test(n))
      print(n);
  }
}
\end{lstlisting}
\end{frame}
\begin{frame}[fragile]
  \frametitle{Primes generator}
  \note{Imagine we want to print successive prime numbers until told to stop}
  \begin{lstlisting}

generator<int> primes() {

  prime_tester tester;
  for (int n = 2; ; ++n)

    if (tester.test(n))
      co_yield n;

}
\end{lstlisting}
\end{frame}

\begin{frame}[fragile]
  \frametitle{Primes generator}
  \begin{itemize}
  \item To use the generator, just iterate over it.
\begin{lstlisting}
for (int n : primes()) {
  cout << n << "\n";
  if (n >= 100) break;
}
\end{lstlisting}
  \item This is nice and clean: \lstinline~primes~ is only concerned with generating
    all the primes and all the logic about what to do with them is external
  \end{itemize}
\end{frame}

\begin{frame}[fragile]
  \frametitle{Primes generator}
\begin{lstlisting}
generator<int> primes() {
  prime_tester tester;
  for (int n = 2; ; ++n)
    if (tester.test(n))
      co_yield n;
  }
}
\end{lstlisting}
  \begin{itemize}
  \item \lstinline~co_yield~ is the first of three new keywords
  \item \lstinline~generator<T>~ is a class template with \lstinline~begin()~
    and \lstinline~end()~ members
  \item \lstinline~generator~ is not provided by the TS but you can make it
    yourself
  \end{itemize}
\end{frame}

\begin{frame}[fragile]
  \frametitle{Generator composability}
\begin{lstlisting}
generator<int> ints(int from) {
  for (int i = from; ; ++i)
    co_yield i;
}

template <typename Range, typename Pred>
generator<int> filter(Range range, Pred pred) {
  for (auto&& x : range)
    if (pred(x))
      co_yield x;
}

generator<int> primes() {
  return filter(ints(2), is_prime);
}
\end{lstlisting}
\end{frame}

\note[itemize]
{
\item return type of filter is not generic enough
\item primes is not a coroutine
\item callers can't tell the difference
}

\begin{frame}[fragile,t]
  \frametitle{Example: Reading from a Socket}
  Synchronously:
\begin{lstlisting}
       string  connect_and_read(string host,
                                int port) {
  string result;
  auto socket =          connect(host, port);
  std::array<char, 1024> buffer;
  while (int nread =
                    socket.read(buffer))
    result.append(buffer, nread);
     return result;
}
\end{lstlisting}
\end{frame}
\begin{frame}[fragile,t]
  \frametitle{Example: Reading from a Socket}
  Asynchronously:
\begin{lstlisting}
future<string> connect_and_read(string host,
                                int port) {
  string result;
  auto socket = co_await connect(host, port);
  std::array<char, 1024> buffer;
  while (int nread =
           co_await socket.read(buffer))
    result.append(buffer, nread);
  co_return result;
}
\end{lstlisting}
  \begin{itemize}
  \item Two more new keywords: \lstinline~co_await~ and \lstinline~co_return~
  \item \lstinline~future~ could be \lstinline~std::future~ or any other class that has the
    coroutine customization points implemented
  \end{itemize}
\end{frame}

\begin{frame}[fragile]
  \frametitle{Example: Fetching Primes Asynchronously}
\begin{lstlisting}
future<bool> is_prime(int n);

stream<int> primes() {
  for (int n : ints(2))
    if (co_await is_prime(n))
      co_yield n;
}

bool stop;
for co_await (int n : primes()) {
  cout << n << "\n";
  cin >> stop;
  if (stop) break;
}
\end{lstlisting}
\end{frame}

\end{document}

% Local Variables:
% TeX-command-extra-options: "-shell-escape"
% End:
